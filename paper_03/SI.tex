\PassOptionsToPackage{unicode=true}{hyperref} % options for packages loaded elsewhere
\PassOptionsToPackage{hyphens}{url}
%
\documentclass[]{article}
\usepackage{lmodern}
\usepackage{amssymb,amsmath}
\usepackage{ifxetex,ifluatex}
\usepackage{fixltx2e} % provides \textsubscript
\ifnum 0\ifxetex 1\fi\ifluatex 1\fi=0 % if pdftex
  \usepackage[T1]{fontenc}
  \usepackage[utf8]{inputenc}
  \usepackage{textcomp} % provides euro and other symbols
\else % if luatex or xelatex
  \usepackage{unicode-math}
  \defaultfontfeatures{Ligatures=TeX,Scale=MatchLowercase}
\fi
% use upquote if available, for straight quotes in verbatim environments
\IfFileExists{upquote.sty}{\usepackage{upquote}}{}
% use microtype if available
\IfFileExists{microtype.sty}{%
\usepackage[]{microtype}
\UseMicrotypeSet[protrusion]{basicmath} % disable protrusion for tt fonts
}{}
\IfFileExists{parskip.sty}{%
\usepackage{parskip}
}{% else
\setlength{\parindent}{0pt}
\setlength{\parskip}{6pt plus 2pt minus 1pt}
}
\usepackage{hyperref}
\hypersetup{
            pdfborder={0 0 0},
            breaklinks=true}
\urlstyle{same}  % don't use monospace font for urls
\usepackage{longtable,booktabs}
% Fix footnotes in tables (requires footnote package)
\IfFileExists{footnote.sty}{\usepackage{footnote}\makesavenoteenv{longtable}}{}
\usepackage{graphicx,grffile}
\makeatletter
\def\maxwidth{\ifdim\Gin@nat@width>\linewidth\linewidth\else\Gin@nat@width\fi}
\def\maxheight{\ifdim\Gin@nat@height>\textheight\textheight\else\Gin@nat@height\fi}
\makeatother
% Scale images if necessary, so that they will not overflow the page
% margins by default, and it is still possible to overwrite the defaults
% using explicit options in \includegraphics[width, height, ...]{}
\setkeys{Gin}{width=\maxwidth,height=\maxheight,keepaspectratio}
\setlength{\emergencystretch}{3em}  % prevent overfull lines
\providecommand{\tightlist}{%
  \setlength{\itemsep}{0pt}\setlength{\parskip}{0pt}}
\setcounter{secnumdepth}{0}
% Redefines (sub)paragraphs to behave more like sections
\ifx\paragraph\undefined\else
\let\oldparagraph\paragraph
\renewcommand{\paragraph}[1]{\oldparagraph{#1}\mbox{}}
\fi
\ifx\subparagraph\undefined\else
\let\oldsubparagraph\subparagraph
\renewcommand{\subparagraph}[1]{\oldsubparagraph{#1}\mbox{}}
\fi

% set default figure placement to htbp
\makeatletter
\def\fps@figure{htbp}
\makeatother


\date{}

\begin{document}

\textbf{Modeling Carbon Dioxide Vibrational Frequencies in Ionic
Liquids: II. Spectroscopic Map}

Clyde A. Daly Jr.,\textsuperscript{1} Eric J.
Berquist,\textsuperscript{2,3} Thomas Brinzer,\textsuperscript{2,3} Sean
Garrett-Roe,\textsuperscript{2,3} Daniel S.
Lambrecht,\textsuperscript{2,3} and Steven A.
Corcelli\textsuperscript{1}

\textsuperscript{1} Department of Chemistry and Biochemistry, University
of Notre Dame, 251 Nieuwland Science Hall, Notre Dame, Indiana 46656

\textsuperscript{2} Department of Chemistry, University of Pittsburgh,
219 Parkman Ave., Pittsburgh, Pennsylvania 15260

\textsuperscript{3} Pittsburgh Quantum Institute, University of
Pittsburgh, 3943 O'Hara St., Pittsburgh, Pennsylvania 15260

\textbf{1. Transition Dipole Moment Calculations}

The intensity of a vibrational transition,
\({\widetilde{\nu}}_{\text{if}}\), is related to the dipole moment
matrix element between the two states,
\(\left\langle {\overset{}{\mu}}_{\text{if}} \right\rangle\)

\[I\left( {\widetilde{\nu}}_{\text{if}} \right) = \frac{8\pi^{3}N_{A}}{3\text{hc}\left( 4\pi\epsilon_{0} \right)}{\widetilde{\nu}}_{\text{if}}\left| {\overset{}{\mu}}_{\text{if}} \right|^{2}(N_{i} - N_{f})\]

(1)

where \(N_{A}\) is Avogadro's number, \(N_{k}\) is the number of
particles in the \(k\)th state and
\(\left| {\overset{}{\mu}}_{\text{if}} \right|^{2}\) is the squared norm
of the transition dipole moment (TDM) integral between the two
states.\textsuperscript{1} Because all values in equation 1 are constant
(at a specific temperature) vibrational intensities for particular
transitions are proportional to the squared norm of the TDM vector.
Thus, the central property to calculate in order to evaluate the
strength of the Condon approximation is
\(\left\langle {\overset{}{\mu}}_{\text{if}} \right\rangle\).

We can calculate the matrix elements of the dipole moment operator in a
similar fashion as the bond length matrix elements were calculated in
paper 1. Before, we used the value of the bond length at each grid point
as a representation of the bond length operator. Similarly, we use the
x, y, and z components of the dipole moment at each grid point (reported
by a quantum chemistry program -- in this case,
Q-Chem\textsuperscript{2} -- as the appropriate integral over the entire
charge density) as a representation of the dipole moment operator,

\[\overset{}{\mu} = \sum_{k = 1}^{3}{\overset{}{\mu}\widehat{k}}\]

(2)

where \(\widehat{k}\) is the \(k\)\textsuperscript{th} Cartesian basis
vector. The dipole moment matrix elements are, for a two dimensional
grid,

\[\left\langle {\overset{}{\mu}}_{\text{if}} \right\rangle = \sum_{k = 1}^{3}{\sum_{l = 1}^{N}{\sum_{j = 1}^{N}{\psi_{\text{jl}}^{i}{\overset{}{\mu}}_{\text{jl}}\widehat{k}\psi_{\text{jl}}^{f}}}}\]

(3)

where the \(\psi_{\text{jl}}^{n}\) are the vibrational wavefunctions for
state \(n\) on grid point \((j,l)\) returned by the DVR method. We have
evaluated the accuracy of this method for CO\textsubscript{2} in two
ways. First, we calculate the norm of the TDM integral for the symmetric
and asymmetric stretches of CO\textsubscript{2} in the gas phase and
compare these to experiment.\textsuperscript{3} The results are shown in
Table S1, and the accuracy is excellent.

Next, we evaluated the accuracy of this method for CO\textsubscript{2}
in solution. A previously used method for evaluating the Condon
approximation for vibrational reporters in solution is to (1) optimize
the vibrational subsystem of interest with DFT while freezing all other
degrees of freedom, (2) calculate the harmonic vibrational frequency and
intensity for the vibrational subsystem using the same DFT method, then
(3) repeat this for many statistically independent snapshots of the
reporter in solution.\textsuperscript{4} This process was completed for
25 snapshots of CO\textsubscript{2} in IL solution for the asymmetric
stretch. DVR asymmetric stretch frequencies and TDMs were also
calculated for the 25 optimized (post step 1) snapshots. In order to
facilitate comparison, the square roots of the intensities were taken.
The resulting values and the TDMs were divided by their respective gas
phase values. These values are plotted against each other in figure S1.
The agreement between the two methods is excellent (R = 0.994). This new
method has the advantage of being essentially computationally free to
perform anytime a DVR calculation has already been done. Due to the
possibility of parallelization, DVR calculations can be much more
computationally inexpensive than regular vibrational frequency
calculations.

\textbf{References}

(1) Carbonnière, P.; Dargelos, A.; Pouchan, C. The VCI-P Code: An
Iterative Variation-Perturbation Scheme for Efficient Computations of
Anharmonic Vibrational Levels and IR Intensities of Polyatomic
Molecules. \emph{Theor. Chem. Acc.} \textbf{2010}, \emph{125} (3--6),
543--554.

(2) Shao, Y.; Gan, Z.; Epifanovsky, E.; Gilbert, A. T. B.; Wormit, M.;
Kussmann, J.; Lange, A. W.; Behn, A.; Deng, J.; Feng, X.; et al.
Advances in Molecular Quantum Chemistry Contained in the Q-Chem 4
Program Package. \emph{Mol. Phys.} \textbf{2015}, \emph{113} (2),
184--215.

(3) Downing, H. D.; Krohn, B. J.; Hunt, R. H. Coriolis Intensity
Perturbations in \(\Pi - \Sigma\) Bands of CO\textsubscript{2}. \emph{J.
Mol. Spectrosc.} \textbf{1975}, \emph{55}, 66--80.

(4) Schmidt, J. R.; Corcelli, S. A.; Skinner, J. L. Pronounced
Non-Condon Effects in the Ultrafast Infrared Spectroscopy of Water.
\emph{J. Chem. Phys.} \textbf{2005}, \emph{123} (4), 044513.

\textbf{\\
}

\textbf{Table S1}. Transition dipole moments for gas phase stretching
modes of CO\textsubscript{2}.

\begin{longtable}[]{@{}lll@{}}
\toprule
Mode & DVR (D) & Experiment (D)\textsuperscript{3}\tabularnewline
\midrule
\endhead
\(\omega_{s}\) & \(1.1\  \times \ 10^{- 13}\) & \(0.0\)\tabularnewline
\(\omega_{a}\) & \(3.4\  \times \ 10^{- 1}\) &
\(3.3\  \times \ 10^{- 1}\)\tabularnewline
\bottomrule
\end{longtable}

% \includegraphics[width=4.31344in,height=4.20268in]{media/image1.emf}

\textbf{Figure S1.} Normalized transition dipole moment for the
asymmetric stretch of CO\textsubscript{2} as calculated by a quantum
chemistry program and as calculated by the DVR method (blue dots). The
black line is the best fit line, \(y = 0.97x + 0.07\). The correlation
coefficient is 0.994.

\end{document}
