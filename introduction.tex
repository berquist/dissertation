\documentclass[%
class = book,%
crop = false,%
float = true,%
multi = true,%
preview = false,%
]{standalone}
\onlyifstandalone{\usepackage{amsmath}}
\onlyifstandalone{\usepackage{xcolor}}
\usepackage{cancel}
% https://tex.stackexchange.com/q/105902/94717
% \newcommand\Ccancel[2][black]{\renewcommand\CancelColor{\color{#1}}\cancel{#2}}
\newcommand\Ccancelto[3][black]{\renewcommand\CancelColor{\color{#1}}\cancelto{#2}{#3}}
\begin{document}
\chapter{Introduction}
\label{ch:introduction}

\section{Introduction to the thesis}
What is the unifying theme?

The unifying theme is the contribution of specific intermolecular interactions to spectroscopic response.

The contributions of specific intermolecular interactions are in the language of energy decomposition analysis using absolutely localized molecular orbitals, abbreviated as ALMO-EDA.

Intro/background to ALMO-EDA is given in sec 4.2

In chapters blah, blah, blah, we present applications of response theory to calculating spectroscopic properties, specifically vibrational frequencies and dipole polarizabilities.

The remainder of this introduction will cover the basic language of molecular response theory and its connection to macroscopic spectroscopic observables.

\section{Connection between macroscopic and microscopic}

maybe subsection of ADT

\section{Analytic derivative theory}

\begin{itemize}
\item start from MacLaurin series expansion of the perturbation
\item identify each derivative as corresponding to a molecular property
\item example tables of molecular properties (not until after frequency-dependent perturbations)
\item however examples for Hessian, polarizability (static) are ok
\item some explanation of why this is insufficient for frequency-dependent perturbations
\item finite difference?
\end{itemize}

Sources: Toulouse; Handy has expressions for derivatives leading to integrals

Two possible ways to perform the derivation:

\begin{enumerate}
\item from series expansion of the energy with respect to one or more perturbations
\item from a phenomenological Hamiltonian (sort of like quantizing a classical expression)
\end{enumerate}

\subsection{Phenomenological approach}

For a one-dimensional spring connecting a ball to a fixed object, Hooke's law

\begin{equation}
  F = -k x
  \label{eq:hooke_1d}
\end{equation}

Generalization to 3D space

\begin{equation}
  \begin{bmatrix}
    F_{1} \\ F_{2} \\ F_{3}
  \end{bmatrix}
  = -
  \begin{bmatrix}
    k_{11} & k_{12} & k_{13} \\
    k_{21} & k_{22} & k_{23} \\
    k_{31} & k_{32} & k_{33}
  \end{bmatrix}
  \begin{bmatrix}
    x_{1} \\ x_{2} \\ x_{3}
  \end{bmatrix}
\end{equation}

Can also be made 3N-dimensional, describes the forces on atoms given their relative positions \(\mathbf{x}\) and the ``stiffness'' of their connectivity \(\mathbf{k}\)

\begin{equation}
  F_{i} = - \sum_{j} k_{ij} x_{j}
\end{equation}

Rewrite in Einstein summation notation

\begin{equation}
  F_{i} = -k_{ij} x_{j}
  \label{eq:force_hooke_einstein}
\end{equation}

We also know that the force is related to the energy. In one dimension

\begin{align}
  F &= - \nabla U \\
  &= \frac{\partial E}{\partial x}
\end{align}

In arbitrary dimensions

\begin{equation}
  F_{i} = \frac{\partial E}{\partial x_{i}}
  \label{eq:force_partial_derivative}
\end{equation}

Equating the \(F_{i}\) in \eqref{eq:force_hooke_einstein} and \eqref{eq:force_partial_derivative} gives

\begin{equation}
  -k_{ij} x_{j} = \frac{\partial E}{\partial x_{i}}
\end{equation}

To solve for the stiffness coefficients, take the partial derivative of both sides with respect to \(x_{j}\), using the product rule on the left hand side

\begin{align}
  \left( \frac{\partial}{\partial x_{j}} \right) \left( -k_{ij} x_{j} \right) &= \left( \frac{\partial}{\partial x_{j}} \right) \left( \frac{\partial E}{\partial x_{i}} \right) \\
  \Ccancelto[red]{0}{\left[ \left( \frac{\partial}{\partial x_{j}} \right) \left( -k_{ij} \right) \right]} x_{j} - k_{ij} \Ccancelto[green]{1}{\left[ \left( \frac{\partial}{\partial x_{j}} \right) x_{j} \right]} &= \frac{\partial^{2} E}{\partial x_{j} \partial x_{i}} \\
  -k_{ij} &= \frac{\partial^{2} E}{\partial x_{j} \partial x_{i}}
\end{align}

This tells that the internal stiffness is related to the second derivative of the energy with respect to nuclear coordinate displacements. The internal stiffness is actually the molecular Hessian.

Wikipedia: symmetry of second derivatives, Euler's (interchange) theorem

\begin{equation}
  \frac{\partial^{2} E}{\partial x_{j} \partial x_{i}} = \frac{\partial^{2} E}{\partial x_{i} \partial x_{j}}
\end{equation}

Comment on which energy expression is being differentiated: it is the energy expression for the chosen method (HF, \(\omega\)B97M-V, MP2, CCSD(T), \dots)

A similar derivation holds for the dipole polarizability, \(\alpha\), which is the ratio of the induced dipole moment \(\mu\) of a system to the electric field \(E\) that produces this dipole moment. Although the electric field is usually written as \(E\) or \(F\), here \(\epsilon\) is used to disambiguate it from the geometric force in \eqref{eq:hooke_1d}. Both \(\mu\) and \(\epsilon\) are 3D vector quantities

\begin{equation}
  \mu = \alpha \epsilon
\end{equation}

\begin{equation}
  \begin{bmatrix}
    \mu_{1} \\ \mu_{2} \\ \mu_{3}
  \end{bmatrix}
  = -
  \begin{bmatrix}
    \alpha_{11} & \alpha_{12} & \alpha_{13} \\
    \alpha_{21} & \alpha_{22} & \alpha_{23} \\
    \alpha_{31} & \alpha_{32} & \alpha_{33}
  \end{bmatrix}
  \begin{bmatrix}
    \epsilon_{1} \\ \epsilon_{2} \\ \epsilon_{3}
  \end{bmatrix}
\end{equation}

Another definition of the molecular dipole moment induced by an external (applied) electric field is

\begin{equation}
  \mu_{i} = \frac{\partial E}{\partial \epsilon_{i}}
\end{equation}

\begin{equation}
  \alpha_{ij} \epsilon_{j} = \frac{\partial E}{\partial \epsilon_{i}}
\end{equation}

\begin{align}
  \left( \frac{\partial}{\partial \epsilon_{j}} \right) \left( -\alpha_{ij} \epsilon_{j} \right) &= \left( \frac{\partial}{\partial \epsilon_{j}} \right) \left( \frac{\partial E}{\partial \epsilon_{i}} \right) \\
  -\Ccancelto[red]{0}{\left( \frac{\partial \alpha_{ij}}{\partial \epsilon_{j}} \right)} \epsilon_{j} - \alpha_{ij} \Ccancelto[green]{1}{\left( \frac{\partial \epsilon_{j}}{\partial \epsilon_{j}} \right)} &= \frac{\partial^{2} E}{\partial \epsilon_{j} \partial \epsilon_{i}} \\
  -\alpha_{ij} &= \frac{\partial^{2} E}{\partial \epsilon_{j} \partial \epsilon_{i}}
\end{align}

\section{Frequency-dependent response}

\begin{itemize}
\item Again, unclear on how ADT works in the presence of time-dependent or oscillating fields
\item Two approaches: quasi-energy derivatives and polarization propagator
\item In the static limit (\(\omega \rightarrow 0, t \rightarrow \infty\) (?)), equivalent to ADT, this is a more general derivation
\item This is a one-particle approximation, that is, assume a perturbation can be described as a linear combination of single-particle excitations and deexcitations
\item Two-particle approximation is SOPPA
\end{itemize}

Sources: Toulouse; McWeeny, summer school book draft, Karna/Dupuis
\end{document}
