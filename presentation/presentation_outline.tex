\documentclass[xcolor=usenames,dvipsnames,svgnames]{beamer}

\RequirePackage{xparse}

%% https://www.quora.com/Presentations-What-are-the-best-beamer-themes
\setbeamertemplate{frametitle}
  {\begin{centering}\smallskip
   \insertframetitle\par
   \smallskip\end{centering}}
\setbeamertemplate{itemize item}{\(\bullet\)}
\setbeamertemplate{navigation symbols}{}
\setbeamertemplate{footline}[text line]{%
    \hfill\strut{%
        \scriptsize\sf\color{black!60}%
        \quad\insertframenumber
    }%
    \hfill
}

\definecolor{NoteBrown}{RGB}{152,101,47}
\definecolor{NoteBlue}{RGB}{31,91,255}
\definecolor{NoteGreen}{RGB}{62,195,74}
\definecolor{NoteRed}{RGB}{191,0,0}
\definecolor{NotePurple}{RGB}{151,54,208}
\definecolor{NoteDarkGreen}{RGB}{0,121,62}
\definecolor{NoteGrey}{RGB}{150,150,150}
\definecolor{NoteOrange}{RGB}{255,155,0}

% Define some colors:
\definecolor{PittBlue}{RGB}{0,43,94}
\definecolor{PittGold}{RGB}{197,168,118}
\definecolor{DarkFern}{HTML}{407428}
\definecolor{DarkCharcoal}{HTML}{4D4944}
\colorlet{Fern}{DarkFern!85!white}
\colorlet{Charcoal}{DarkCharcoal!85!white}
\colorlet{LightCharcoal}{Charcoal!50!white}
\colorlet{AlertColor}{orange!80!black}
\colorlet{DarkRed}{red!70!black}
\colorlet{DarkBlue}{blue!70!black}
\colorlet{DarkGreen}{green!70!black}
% Use the colors:
\setbeamercolor{title}{fg=PittBlue}
\setbeamercolor{frametitle}{fg=PittBlue}
\setbeamercolor{normal text}{fg=DarkCharcoal}
\setbeamercolor{block title}{fg=black,bg=Fern!25!white}
\setbeamercolor{block body}{fg=black,bg=Fern!25!white}
\setbeamercolor{alerted text}{fg=AlertColor}
\setbeamercolor{itemize item}{fg=Charcoal}

\usepackage[%
    backend    = biber,%
    style      = chem-acs,%
    autocite   = superscript,%
    backref    = true,%
    biblabel   = brackets,%
    doi        = true,%
    minnames   = 1,%
    maxnames   = 10,%
]{biblatex}
\addbibresource{../library.bib}
\addbibresource{../library2.bib}
\addbibresource{../paper_04/paper.bib}
\addbibresource{../paper_05/psi4numpy.bib}
\addbibresource{../paper_05/tutorial.bib}
\addbibresource{../5746059.bib}
\addbibresource{../5773968.bib}
\addbibresource{../in_preparation.bib}
\addbibresource{./presentation.bib}

\AtEveryCitekey{\iffootnote{\color{PittBlue}\tiny}{}}
\NewDocumentCommand{\nmfootfullcite}{m}{%
    \AtNextCite{%
        \let\thefootnote\relax
        \let\mkbibfootnote\mkbibfootnotetext
    }{%
        \footfullcite{#1}
    }
}

\RequirePackage{appendixnumberbeamer}
\RequirePackage{booktabs}
\RequirePackage{braket}
\RequirePackage{caption}
\RequirePackage{graphicx}
\RequirePackage{microtype}
\RequirePackage{minted}
\RequirePackage[version=4]{mhchem}
\newcommand{\arlidimer}{\ce{Ar\bond{....}Li+}}
\RequirePackage[%
    separate-uncertainty = true,%
    multi-part-units = single,%
    range-units = single,%
    retain-explicit-plus = true,%
]{siunitx}
\RequirePackage{subcaption}
\RequirePackage{tabulary}

\NewDocumentCommand{\vect}{m}{%
        \ensuremath{\boldsymbol{\mathbf{#1}}}%
}%

% \newcommand*{\phnum}[2][]{%
%   \begingroup
%     \protected\def\X{\phantom{0}}%
%     \num[
%       input-digits=0123456789\X,%
%       #1%
%     ]{#2}%
%   \endgroup
% }

% \newcommand\pfour{\textsc{Psi4}}
\newcommand\pfour{Psi4}
% \newcommand\pfn{\textsc{Psi4NumPy}}
\newcommand\pfn{Psi4NumPy}

\title{Decomposition of Intermolecular Interactions in \textit{Ab Initio} Spectroscopy}
\author{Eric Berquist}
% \institute[Pitt]{Lambrecht Research Group \\ Dissertation Defense \\ \vspace{6pt} \includegraphics[width=1in]{./figures/pitt_logo.pdf}}
\institute{\includegraphics[width=1in]{./figures/pitt_logo.pdf}}
\date{March 23th, 2018}

\begin{document}

\frame{
  \titlepage
}

\frame{
  \frametitle{Outline}
  \tableofcontents
}

\section{Decomposition of harmonic vibrational frequencies}

\section{Decomposition of general linear response properties}

\begin{frame}
  \frametitle{Numerical benchmarks using static electric dipole polarizabilities}
\end{frame}

\begin{frame}
  \frametitle{LR(MI) analytic static polarizabilities are quantitatively correct compared to finite difference results at short range}
  \note{This shows the correctness of our implementation}
\end{frame}

\begin{frame}
  \frametitle{LR(MI) polarizabilities have physically-correct long-range asymptotic behavior}
\end{frame}

\begin{frame}
  \frametitle{libresponse is available as an open-source library}
\end{frame}

\section{Reference and tutorial implementations of quantum chemical methods}

\begin{frame}
  \frametitle{\pfn{} provides a modern ecosystem for quantum chemical method development}
  \centering
  \includegraphics[width=\linewidth,keepaspectratio]{./figures/psi4numpy_ecosystem2.pdf}
\end{frame}

\begin{frame}
  \frametitle{\pfn{} is an ideal platform for teaching about \textit{ab initio} spectroscopy}
\end{frame}

\begin{frame}
  \frametitle{Jupyter Notebooks allow mixing of concepts, equations, and code in an exploratory environment}
\end{frame}

\begin{frame}
  \frametitle{\pfn{} is welcoming to new contributors}
\end{frame}

\begin{frame}
  \frametitle{Future goals}
\end{frame}

\begin{frame}
  \frametitle{Conclusions}
\end{frame}

\begin{frame}
  \frametitle{Acknowledgements}
\end{frame}

\begin{frame}
  \frametitle{Thank you}
\end{frame}

\appendix

\begin{frame}
  \frametitle{LR(MI) is BSSE-free by construction}
\end{frame}

\end{document}
