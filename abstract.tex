% no more than 350 words, use C-c C-w c in the abstract region
\begin{abstract}
  [TODO remove references, abstract must be meaningful without citations]
  Spectroscopy, the molecular response to electromagnetic radiation of different wavelengths, is one of the most powerful experimental tools for interrogating a molecule's interactions with its environment. However, both the time-averaged and time-resolved pictures are not enough to obtain exact structural information, requiring the construction of molecular models for correlating geometries with spectroscopic response. [TODO mention the effect of intermolecular interactions on spectroscopic response]
  % paper 1
  Quantum chemical calculations on small clusters qualitatively reproduced the experimental ordering for \ce{CO2}'s \(\nu_3\) peak position when dissolved in a series of ionic liquids when varying the anion. To uncover the physical origin of the \(\nu_3\) shift, the language of decomposition analysis, specifically ALMO-EDA, was translated from energies to vibrational frequencies. We found that geometric distortion of the \ce{CO2} in the ionic liquid environment, as a result of charge transfer (CT) from the anion into the \ce{CO2}, is the driving force for differentiating the \ce{CO2} \(\nu_3\) shift in different ionic liquid anions.
  % papers 2/3
  [mention ``geometry mechanism'' and ``curvature mechanism''], [mention comparison between ALMO-EDA and SAPT]
  % paper 4
  The decomposition developed in Ref.~\parencite{Brinzer2015} and further applied in Ref.~\parencite{Berquist2017} presented the first application of an EDA outside the energy realm into molecular properties. However, the method used for calculating the CT contribution to the curvature mechanism is not generally applicable to arbitrary response properties. A reformulation of the linear response equations into
  % paper 5
  The fully-analytic decomposition presented in Ref.~\parencite{Berquist2018}
\end{abstract}

% Local variables:
% eval: (wc-mode t)
% End:
