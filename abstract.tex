\documentclass[%
  class = article,%
  crop = false,%
  float = true,%
  multi = true,%
  preview = false,%
]{standalone}
\usepackage[version=4]{mhchem}
\newcommand{\arlidimer}{\ce{Ar\bond{....}Li+}}
% no more than 350 words, use C-c C-w c in the abstract region
\begin{document}
Spectroscopy, the molecular response to electromagnetic radiation of different wavelengths, is one of the most powerful experimental tools for interrogating a molecule's structure and dynamics as it interacts with its environment. However, relating a spectroscopic signature to a molecular picture relies on sophisticated computational approaches, which offer a wealth of methods for identifying structures, intermolecular interactions, and their correlation with spectroscopic response. This thesis focuses on the how to correlate a molecule's structure and interactions with its environment via \textit{ab initio} calculation of spectroscopic parameters.

To build a molecular picture of \ce{CO2} dynamics in ionic liquids (ILs), quantum chemical calculations on small clusters qualitatively reproduced the experimental ordering for \ce{CO2}'s asymmetric vibrational stretch (\(\nu_3\)) peak position which shifts when dissolved in a series of ILs with varying anions. To uncover the physical origin of the shift, the language of decomposition analysis based on absolutely localized molecular orbitals (ALMO-EDA) was translated from energies to vibrational frequencies. Geometric distortion of \ce{CO2}, as a result of charge transfer (CT) from the anion into the \ce{CO2}, is the driving force for differentiating the \ce{CO2} \(\nu_3\) shift in different IL anions.

After validating these simple models, we further decomposed the CT contribution into geometry and curvature mechanisms, finding that CT is a significant contributor in both the geometry optimization and frequency calculation steps. A comparison between ALMO-EDA and symmetry-adapted perturbation theory (SAPT) showed that while dispersion dominates the binding energy, excellent correlation between both total interaction energies and individual components for ALMO-EDA and SAPT validates the use of DFT, enabling the construction of a semiempirical spectroscopic map.

This decomposition presented one of the first large-scale applications of an EDA outside the energy realm into molecular properties; however, it is not generally applicable to arbitrary perturbations. A reformulation of the canonical linear response equations for use with ALMOs provides a direct connection between EDA terms and their corresponding contribution to spectra. Results for \arlidimer{} polarizabilities show that allowing CT is equally important in both the underlying ground-state wavefunction and the response calculation, and should not be confused with basis set superposition error.
\end{document}

% Local variables:
% eval: (wc-mode t)
% End:
