\documentclass[%
  class = article,%
  crop = false,%
  float = true,%
  multi = true,%
  preview = false,%
]{standalone}
\onlyifstandalone{\usepackage[version=4]{mhchem}}
\onlyifstandalone{\usepackage{setspace}}
\onlyifstandalone{\doublespacing}
\onlyifstandalone{\title{Thesis Abstract}}
\onlyifstandalone{\author{Eric Berquist}}
\onlyifstandalone{\date{March 23rd, 2018}}
% no more than 350 words, use C-c C-w c in the abstract region
\begin{document}
\onlyifstandalone{\maketitle}
Spectroscopy, the response of matter to electromagnetic radiation of different wavelengths, is a powerful experimental tool for interrogating a molecule's structure and dynamics as it interacts with its environment. However, relating a spectroscopic signature to a molecular picture relies on sophisticated computational approaches in order to identify structures, intermolecular interactions, and their correlation with spectroscopic response. This thesis focuses on the question of how to correlate a molecule's structure and interactions with its environment via the \textit{ab initio} calculation of spectroscopic parameters.

To build a molecular picture of \ce{CO2} dynamics in ionic liquids (ILs), I performed quantum chemical calculations on small gas-phase \ce{CO2}-IL clusters, qualitatively reproducing the experimental ordering for \ce{CO2}'s asymmetric vibrational stretch (\(\nu_3\)) peak position as a function of the anion. To uncover the physical origin of the shift, the language of decomposition analysis based on absolutely localized molecular orbitals (ALMO-EDA) was translated from energies to vibrational frequencies. Geometric distortion of \ce{CO2}, as a result of charge transfer (CT) from the anion into the \ce{CO2}, is the driving force for differentiating the \ce{CO2} \(\nu_3\) shift in different IL anions.

After validating these simple models, I further decomposed the CT contribution into equilibrium structure and potential energy surface curvature mechanisms, finding that CT is a significant contributor in both the geometry optimization and frequency calculation steps. Comparing ALMO-EDA and symmetry-adapted perturbation theory (SAPT) showed that while dispersion dominates the binding energy, excellent correlation between both the total interaction energies and individual components validates the use of DFT-based ALMO-EDA over wavefunction-based SAPT, enabling the construction of a semiempirical spectroscopic map.

This decomposition presented one of the first large-scale applications of an EDA outside the energy realm into molecular properties; however, it is not generally applicable to arbitrary perturbations. I reformulated the canonical linear response equations for use with ALMOs to provide a direct connection between EDA terms and their corresponding contribution to spectra. Test calculations indicate that allowing CT is equally important in both the underlying ground-state wavefunction and the response calculations and should not be confused with basis set superposition error.
\end{document}

% Local variables:
% eval: (wc-mode t)
% End:
