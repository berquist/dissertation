\documentclass{article}

\usepackage{booktabs}
\usepackage{csvsimple}
\usepackage[margin=1in]{geometry}
\raggedbottom
\usepackage{longtable}
\usepackage{microtype}
\usepackage{pdflscape}
\usepackage{siunitx}
\DeclareSIUnit[number-unit-product = {\,}]\cal{cal}
\DeclareSIUnit\kcal{\kilo\cal}

\begin{document}
% \begin{landscape}
%   \csvreader[%
%   respect underscore=true,
%   % longtable=lrSc*{13}{S[table-auto-round=true,table-format=2.4]},
%   % longtable=lrcc*{13}{c},
%   longtable=lrcc*{13}{S[table-auto-round=true,table-format=2.4]},
%   table head=\caption{SAPT0 interaction energies for different MD snapshots using the 6-31G** basis set. All energies are in units of \si{\kcal\per\mole}.}\label{tab:si_015}\\\toprule & snapshot \# & weight & SAPT basis & {\(E_{\text{el}}^{(10)}\)} & {\(E_{\text{exch}}^{(10)}\)} & {``frz''} & {\(E_{\text{ind}}^{(20)}\)} & {\(E_{\text{ind-exch}}^{(20)}\)} & {\(\delta_{\text{HF}}\)} & {``pol''} & {``frz + pol''} & {\(E_{\text{disp}}^{(20)}\)} & {\(E_{\text{disp-exch}}^{(20)}\)} & {``disp''} & {total} & {\(E_{\text{CT}}\)} \\ \midrule,
%   table foot=\bottomrule,%
%   ]{jp6b09489_si_015_pople.csv}{}{\csvlinetotablerow}
% \end{landscape}

% Don't use this; rather, separate the individual data points from the averages.
% \begin{landscape}
%   \csvreader[%
%   respect underscore=true,
%   % longtable=lrSc*{13}{S[table-auto-round=true,table-format=2.4]},
%   % longtable=lrcc*{13}{c},
%   longtable=lrcc*{13}{S[table-auto-round=true,table-format=2.4]},
%   table head=\caption{SAPT0 interaction energies for different MD snapshots using the jun-cc-pVTZ basis set. All energies are in units of \si{\kcal\per\mole}.}\label{tab:si_015}\\\toprule & snapshot \# & weight & SAPT basis & {\(E_{\text{el}}^{(10)}\)} & {\(E_{\text{exch}}^{(10)}\)} & {``frz''} & {\(E_{\text{ind}}^{(20)}\)} & {\(E_{\text{ind-exch}}^{(20)}\)} & {\(\delta_{\text{HF}}\)} & {``pol''} & {``frz + pol''} & {\(E_{\text{disp}}^{(20)}\)} & {\(E_{\text{disp-exch}}^{(20)}\)} & {``disp''} & {total} & {\(E_{\text{CT}}\)} \\ \midrule,
%   table foot=\bottomrule,%
%   ]{jp6b09489_si_015_dunning.csv}{}{\csvlinetotablerow}
% \end{landscape}

\begin{landscape}
  \footnotesize
  \csvreader[%
  respect underscore=true,
  longtable=cSc*{13}{S[table-auto-round=true,table-format=2.4]},
  table head=\caption{SAPT0 interaction energies for different MD snapshots using the jun-cc-pVTZ basis set. All energies are in units of \si{\kcal\per\mole}.}\label{tab:si_015}\\\toprule snapshot \# & {weight} & SAPT basis & {\(E_{\text{el}}^{(10)}\)} & {\(E_{\text{exch}}^{(10)}\)} & {``frz''} & {\(E_{\text{ind}}^{(20)}\)} & {\(E_{\text{ind-exch}}^{(20)}\)} & {\(\delta_{\text{HF}}\)} & {``pol''} & {``frz + pol''} & {\(E_{\text{disp}}^{(20)}\)} & {\(E_{\text{disp-exch}}^{(20)}\)} & {``disp''} & {total} & {\(E_{\text{CT}}\)} \\ \midrule,
  table foot=\bottomrule,%
  ]{jp6b09489_si_015_dunning_cleaned_data.csv}{}{\csvlinetotablerow}  
\end{landscape}

% average (unweighted) [monomer],average (weighted) [monomer],100 * fraction of total (unweighted) [monomer],100 * fraction of total (weighted) [monomer],average (unweighted) [dimer],average (weighted) [dimer],100 * fraction of total (unweighted) [dimer],100 * fraction of total (weighted) [dimer]

\begin{landscape}
  \footnotesize
  \csvreader[%
  respect underscore=true,
  longtable=c*{8}{S[table-auto-round=true,table-format=3.2]},
  table head=\caption{SAPT0 interaction energies for different MD snapshots using the jun-cc-pVTZ basis set. All energies are in units of \si{\kcal\per\mole}.}\label{tab:si_015}\\\toprule SAPT component & {average (unweighted) [monomer]} & {average (weighted) [monomer]} & {100 * fraction of total (unweighted) [monomer]} & {100 * fraction of total (weighted) [monomer]} & {average (unweighted) [dimer]} & {average (weighted) [dimer]} & {100 * fraction of total (unweighted) [dimer]} & {100 * fraction of total (weighted) [dimer]} \\ \midrule,
  table foot=\bottomrule,%
  ]{jp6b09489_si_015_dunning_cleaned_summary.csv}{}{\csvlinetotablerow}  
\end{landscape}
\end{document}
