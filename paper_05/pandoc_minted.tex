\PassOptionsToPackage{unicode=true}{hyperref} % options for packages loaded elsewhere
\PassOptionsToPackage{hyphens}{url}
%
\documentclass[]{article}
\usepackage{lmodern}
\usepackage{amssymb,amsmath}
\usepackage{ifxetex,ifluatex}
\usepackage{fixltx2e} % provides \textsubscript
\ifnum 0\ifxetex 1\fi\ifluatex 1\fi=0 % if pdftex
  \usepackage[T1]{fontenc}
  \usepackage[utf8]{inputenc}
  \usepackage{textcomp} % provides euro and other symbols
\else % if luatex or xelatex
  \usepackage{unicode-math}
  \defaultfontfeatures{Ligatures=TeX,Scale=MatchLowercase}
\fi
% use upquote if available, for straight quotes in verbatim environments
\IfFileExists{upquote.sty}{\usepackage{upquote}}{}
% use microtype if available
\IfFileExists{microtype.sty}{%
\usepackage[]{microtype}
\UseMicrotypeSet[protrusion]{basicmath} % disable protrusion for tt fonts
}{}
\IfFileExists{parskip.sty}{%
\usepackage{parskip}
}{% else
\setlength{\parindent}{0pt}
\setlength{\parskip}{6pt plus 2pt minus 1pt}
}
\usepackage{hyperref}
\hypersetup{
            pdfborder={0 0 0},
            breaklinks=true}
\urlstyle{same}  % don't use monospace font for urls
\usepackage[outputdir=.cache/latex]{minted}
\setlength{\emergencystretch}{3em}  % prevent overfull lines
\providecommand{\tightlist}{%
  \setlength{\itemsep}{0pt}\setlength{\parskip}{0pt}}
\setcounter{secnumdepth}{0}
% Redefines (sub)paragraphs to behave more like sections
\ifx\paragraph\undefined\else
\let\oldparagraph\paragraph
\renewcommand{\paragraph}[1]{\oldparagraph{#1}\mbox{}}
\fi
\ifx\subparagraph\undefined\else
\let\oldsubparagraph\subparagraph
\renewcommand{\subparagraph}[1]{\oldsubparagraph{#1}\mbox{}}
\fi

% set default figure placement to htbp
\makeatletter
\def\fps@figure{htbp}
\makeatother


\date{}

\begin{document}

\begin{minted}[]{python}
"""Tutorial: SCF first hyperpolarizability"""

__author__    = "Eric J. Berquist"
__credit__    = ["Eric J. Berquist"]

__copyright__ = "(c) 2014-2017, The Psi4NumPy Developers"
__license__   = "BSD-3-Clause"
__date__      = "2017-12-19"
\end{minted}

\hypertarget{scf-first-hyperpolarizability}{%
\section{SCF First
Hyperpolarizability}\label{scf-first-hyperpolarizability}}

\hypertarget{introduction}{%
\subsection{Introduction}\label{introduction}}

In Tutorial 6a, the calculation of linear response properties from
analytic derivative theory is presented, the foundation of which are the
coupled-perturbed Hartree-Fock (CPHF) or coupled-perturbed
self-consistent field (CPSCF) equations. Starting from analytic
derivative theory provides a convenient physical picture: how does the
total energy of a system change under the influence of one or more
internal or external perturbations? Continuing the case of an external
electric field, the total energy of a system can be represented with a
series expansion:

\[
E(\mathbf{E}) = \sum_{n=0}^{\infty} \frac{1}{n!}E^{(n)}(\mathbf{a})\cdot(\mathbf{E}-\mathbf{a})^{n},
\]

where the electric field is \(\mathbf{E} = \vec{E} = (E_x,E_y,E_z)\) and
\(\mathbf{a}\) is the expansion point. In practice, we always expand
around \(\mathbf{a} = \mathbf{0}\), so it is a Maclaurin series:

\[
E(\mathbf{E}) = \sum_{n=0}^{\infty} \frac{1}{n!}\mathbf{E}^{(n)}(\mathbf{0})\cdot\mathbf{E}^{n}.
\]

Expanding the above to the first 4 explicit terms gives

\[
E(\mathbf{E}) \approx E^{(0)}(\mathbf{0}) + \mathbf{E}^{(1)}(\mathbf{0})\cdot\mathbf{E} + \frac{1}{2}\mathbf{E}^{(2)}(\mathbf{0})\cdot\mathbf{E}^{2} + \frac{1}{6}\mathbf{E}^{(3)}(\mathbf{0})\cdot\mathbf{E}^{3},
\]

where we identify

\begin{align}
E^{(0)} &\rightarrow \textrm{the unperturbed ground-state energy} \\
E_{a}^{(1)} &\rightarrow -\mu_{a},\textrm{the dipole moment} \\
E_{ab}^{(2)} &\rightarrow -\alpha_{ab},\textrm{the polarizability} \\
E_{abc}^{(3)} &\rightarrow -\beta_{abc},\textrm{the first hyperpolarizability}
\end{align}

The first hyperpolarizability is the leading-order term that describes
the \emph{nonlinear} response of a system to an external electric field.
Each term in the series expansion increases the rank of the coefficient
by one: the ground-state energy is a scalar, the dipole is a length 3
vector, the polarizability is a 3-by-3 matrix, and the first
hyperpolarizability is a 3-by-3-by-3 tensor.

Translated into the language of analytic derivative theory, is it
represented as
\(\beta_{abc} = \left.\frac{\partial^{3} E}{\partial E_a \partial E_b \partial E_c}\right|_{\mathbf{E}=\mathbf{0}}\),
though it is not yet clear how to take derivatives of the energy beyond
what is presented in tutorial 6a. Additionally, nothing has been stated
about time dependence; everything to this point has been the static
case, where the strength of fields do not vary with time. We will first
incorporate time dependence, and equations for derivative theory will
result.

\hypertarget{notation}{%
\subsubsection{Notation}\label{notation}}

Before going further, some notational conventions should be mentioned.
When used as field indices, \(a,b,c,\dots \in \{x,y,z\}\), the three
Cartesian directions.

For matrix indices, \(\mu,\nu,\lambda,\sigma,\dots\) label atomic
orbitals (AOs)/basis functions, \(i,j,k,l,\dots\) label occupied
molecular orbitals (MOs), \(a,b,c,d,\dots\) label unoccupied/virtual
MOs, and \(p,q,r,s,\dots\) label all MOs. Einstein summation is used, so
repeated indices are contracted over.

\hypertarget{derivation}{%
\subsection{Derivation}\label{derivation}}

Again, write the total Hamiltonian as the sum of unperturbed and
perturbed components

\begin{align}
\hat{H}(\mathbf{E},t) &= \hat{H}^{(0)} + \hat{V}(\mathbf{E},t) \\
\hat{V}(\mathbf{E},t) &= -\mathbf{\mu} \cdot \mathbf{E}(e^{\pm i \omega t} + 1)
\end{align}

where part of the external field now oscillates with some characteristic
frequency \(\omega\). This can be incorporated into the time-dependent
Schrodinger equation, which for a stationary state obeys

\begin{align}
\left[ \hat{H}^{(0)} + \hat{V}(\mathbf{E},t) - i\frac{\partial}{\partial t} \right] \psi(t) &= 0, \\
FC - i \frac{\partial}{\partial t} SC &= SC\epsilon, \\
\frac{\partial}{\partial t} C^{\dagger} S C &= 0,
\end{align}

where the full definition of the Fock matrix is

\[
F_{\mu\nu} = h_{\mu\nu} + D_{\lambda\sigma}[2J_{\mu\nu\lambda\sigma} - K_{\mu\nu\lambda\sigma}]
\]

and the density matrix is defined as

\[
D_{\mu\nu} = C_{\mu p}n_{pq}C_{\nu q}^{\dagger},
\]

where the diagonal occupation number matrix \(n_{ii} = 2\) and
\(n_{aa} = 0\) for RHF.

In general, the MO coeffients are perturbation- and time-dependent, but
the basis functions themselves are not. This means that when the series
expansion for the perturbation above is performed on other quantities,
only \(F\), \(C\), \(\epsilon\), and \(D\) are affected. For example,
the Lagrangian multiplier matrix \(\epsilon\) can be expanded as

\[
\epsilon(\mathbf{E}) = \epsilon^0 + E_a\epsilon^a + \frac{1}{2!}E_aE_b\epsilon^{ab} + \frac{1}{3!}E_aE_bE_c\epsilon^{abc} + \cdots
\]

where \(a,b,c,...\in\{x,y,z\}\), and

\begin{align}
\epsilon^{a} &= e^{\pm i \omega t} \epsilon^{a}(\pm\omega) + \epsilon^{a}(0), \\
\epsilon^{ab} &= e^{\pm 2 i \omega t} \epsilon^{ab}(\pm\omega,\pm\omega) + e^{\pm i \omega t} \{ \epsilon^{ab}(0,\pm\omega) + \epsilon^{ab}(\pm\omega,0)\} + \epsilon^{ab}(\pm\omega,\mp\omega) + \epsilon^{ab}(0,0), \\
\epsilon^{abc} &= e^{\pm 3 i \omega t} \epsilon^{abc}(\pm\omega,\pm\omega,\pm\omega) + e^{\pm 2 i \omega t} \{\epsilon^{abc}(0,\pm\omega,\pm\omega) + \epsilon^{abc}(\pm\omega,0,\pm\omega) + \epsilon^{abc}(\pm\omega,\pm\omega,0)\} + e^{\pm i \omega t} \{\epsilon^{abc}(\pm\omega,\pm\omega,\mp\omega) + \epsilon^{abc}(\pm\omega,\mp\omega,\pm\omega) + \epsilon^{abc}(\mp\omega,\pm\omega,\pm\omega)\} + e^{\pm i \omega t} \{\epsilon^{abc}(0,0,\pm\omega) + \epsilon^{abc}(0,\pm\omega,0) + \epsilon^{abc}(\pm\omega,0,0)\} + \{\epsilon^{abc}(0,\pm\omega,\mp\omega) + \epsilon^{abc}(\pm\omega,0,\mp\omega) + \epsilon^{abc}(\pm\omega,\mp\omega,0)\} + \epsilon^{abc}(0,0,0),
\end{align}

showing that each order of the expansion consists of all possible phase
combinations. For the first hyperpolarizability, only quantities with at
most two field indices are required. Each permutationally unique subterm
of the expansion corresponds to a different physical observable:

\begin{itemize}
\item
  \((0) \rightarrow\) static polarizability
  \(\rightarrow \alpha(0;0) = -Tr[H^{a} D^{b}(0)]\)
\item
  \((\pm\omega) \rightarrow\) dynamic polarizability
  \(\rightarrow \alpha(\mp\omega;\pm\omega) = -Tr[H^{a} D^{b}(\pm\omega)]\)
\item
  \((0,0) \rightarrow\) static (first) hyperpolarizability
  \(\rightarrow \beta(0;0,0) = -Tr[H^{a} D^{bc}(0,0)]\)
\item
  \((0,\pm\omega) \rightarrow\) electrooptic Pockels effect (EOPE)
  \(\rightarrow \beta(\mp \omega;0,\pm\omega) = -Tr[H^{a} D^{bc}(0,\pm\omega)]\)
\item
  \((\pm\omega,\pm\omega) \rightarrow\) second harmonic generation (SHG)
  \(\rightarrow \beta(\mp 2\omega;\pm\omega,\pm\omega) = -Tr[H^{a} D^{bc}(\pm\omega,\pm\omega)]\)
\item
  \((\pm\omega,\mp\omega) \rightarrow\) optical rectification
  \(\rightarrow \beta(0;\pm\omega,\mp\omega) = -Tr[H^{a} D^{bc}(\pm\omega,\mp\omega)]\)
\end{itemize}

where each property is calculated as the trace over the AO-basis dipole
matrices \(H^{a}\) with the appropriate perturbed density. The task now
comes down to calculating the necessary perturbed density for the
phenomenon of interest. The second-order densities required for the four
different first hyperpolarizabilities are (Karna eqs. III-2a to III-2d)

\begin{align}
D^{ab}(\pm\omega,\pm\omega) &= C^{ab}(\pm\omega,\pm\omega) n C^{0\dagger} + C^{a}(\pm\omega) n C^{b\dagger}(\mp\omega) + C^{b}(\pm\omega) n C^{a\dagger}(\mp\omega) + C^{0} n C^{ab\dagger}(\mp\omega,\mp\omega), \\
D^{ab}(0,\pm\omega) &= C^{ab}(0,\pm\omega) n C^{0\dagger} + C^{a}(0) n C^{b\dagger}(\mp\omega) + C^{b}(\pm\omega) n C^{a\dagger}(0) + C^{0} n C^{ab\dagger}(0,\mp\omega), \\
D^{ab}(\pm\omega,\mp\omega) &= C^{ab}(\pm\omega,\mp\omega) n C^{0\dagger} + C^{a}(\pm\omega) n C^{b\dagger}(\pm\omega) + C^{b}(\mp\omega) n C^{a\dagger}(\mp\omega) + C^{0} n C^{ab\dagger}(\mp\omega,\pm\omega), \\
D^{ab}(0,0) &= C^{ab}(0,0) n C^{0\dagger} + C^{a}(0) n C^{b\dagger}(0) + C^{b}(0) n C^{a\dagger}(0) + C^{0} n C^{ab\dagger}(0,0).
\end{align}

Already a few important insights about the equations are revealed: Each
perturbation index always carries its respective frequency, and the
positive and negative frequencies are related by the Hermitian adjoint
(except for \(C(-\omega) = -C^{0} U^{\dagger}(+\omega)\), Karna eq. 40).
We also see the appearance of terms like \(C^{ab}\), which will require
\(U^{ab}\) originating from the second-order CPHF. Computationally, this
is undesirable due to the increased number of iterative calculations
that must be performed, so we borrow a trick that most prominently
appears in perturbation theory.

\hypertarget{wigners-2n-1-rule}{%
\subsubsection{\texorpdfstring{Wigner's \(2n + 1\)
rule}{Wigner's 2n + 1 rule}}\label{wigners-2n-1-rule}}

From Schaefer, page 25:

\begin{quote}
When the wavefunction is determined up to the \(n\)th order, the
expectation value (electronic energy) of the the system is resolved,
according to the results of perturbation theory, up to the \((2n+1)\)st
order. This principle is called Wigner's \(2n+1\) theorem {[}29-31{]}.
\end{quote}

\[
\begin{array}{cccc}
\hline
 & \text{CI: MO/CI space} & \text{MCSCF: MO/CI space} & \text{RHF: MO space} \\
\hline
\text{Energy}, E & C_{\mu}^{i} , C_{I} & C_{\mu}^{i}, C_{I} & C_{\mu}^{i} \\
\text{First Derivative}, \frac{\partial E}{\partial a} & U^{a} , C_{I} & C_{\mu}^{i}, C_{I} & C_{\mu}^{i}\\
\text{Second Derivative}, \frac{\partial^{2} E}{\partial a \partial b} & U^{ab} , \frac{\partial C_{I}}{\partial a} & U^{a}, \frac{\partial C_{I}}{\partial a} & U^{a} \\
\text{Third Derivative}, \frac{\partial^{3} E}{\partial a \partial b \partial c} & U^{abc} , \frac{\partial C_{I}}{\partial a} & U^{a}, \frac{\partial C_{I}}{\partial a} & U^{a} \\
\text{Fourth Derivative}, \frac{\partial^{4} E}{\partial a \partial b \partial c \partial d} & U^{abcd} , \frac{\partial^{2} C_{I}}{\partial a \partial b} & U^{ab}, \frac{\partial^{2} C_{I}}{\partial a \partial b} & U^{ab} \\
\text{Fifth Derivative}, \frac{\partial^{5} E}{\partial a \partial b \partial c \partial d \partial e} & U^{abcde} , \frac{\partial^{2} C_{I}}{\partial a \partial b} & U^{ab}, \frac{\partial^{2} C_{I}}{\partial a \partial b} & U^{ab} \\
\hline
\end{array}
\]

Since the first hyperpolarizability is calculated as a third derivative
of the energy, perturbed coefficients with only one field index should
be required. From the above table, we can also see why SCF gradients
(\(\frac{\partial E}{\partial R_A}\), where \(R_A\) is the \(A\)-th
Cartesian component of nucleus \(R\)) avoid the need to solve for \(U\)
matrices.

\hypertarget{final-expressions}{%
\subsection{Final expressions}\label{final-expressions}}

To this point, most work has been in the AO basis, but it is
conceptually easier to work in the MO basis, in particular due to the
use of the \(\epsilon\) equations (Karna eq. 34)

\[
\epsilon^{a}(\pm\omega) = G^{a}(\pm\omega) + \epsilon^{0} U^{a}(\pm\omega) - U^{a}(\pm\omega) \epsilon^{0} \pm \omega U^{a}(\pm\omega),
\]

where the \(G\) matrices are the MO-basis Fock matrices

\[
G^{ab\dots} = C^{0\dagger} F^{ab\dots} C^{0},
\]

and the \(U\) matrices are the MO-basis perturbation parameters

\[
C^{ab\dots} = C^{0} U^{ab\dots},
\]

which will be discussed in the implementation. The final expression for
the static hyperpolarizability is (Karna eq. VII-4)

\[
\beta_{abc}(0; 0, 0) = Tr[n \{  U^{a}(0) G^{b}(0) U^{c}(0) + U^{c}(0) G^{b}(0) U^{a}(0) + U^{b}(0) G^{c}(0) U^{a}(0) + U^{a}(0) G^{c}(0) U^{b}(0) + U^{c}(0) G^{a}(0) U^{b}(0) + U^{b}(0) G^{a}(0) U^{c}(0)\} ] - Tr[n \{ U^{a}(0) U^{c}(0) \epsilon^{b}(0) + U^{c}(0) U^{a}(0) \epsilon^{b}(0) + U^{b}(0) U^{a}(0) \epsilon^{c}(0) + U^{a}(0) U^{b}(0) \epsilon^{c}(0) + U^{c}(0) U^{b}(0) \epsilon^{a}(0) + U^{b}(0) U^{c}(0) \epsilon^{a}(0) \} ].
\]

By noticing that each term corresponds to a unique permutation of the
field indices, it can be rewritten as

\[
\beta_{abc} = Tr\left[ n \sum \mathcal{P}(d,e,f) U^{d} G^{e} U^{f} \right] - Tr\left[ n \sum \mathcal{P}(d,e,f) U^{d} U^{e} \epsilon^{f} \right],
\]

where the permutation indices are initially assigned as
\(d = a, e = b, f = c\). The frequency notation has also been dropped,
since each \(abc\) (and therefore each \(def\)) will always carry the
appropriate field index, making this the most general form of the first
hyperpolarizability. If the indices \(abc\) are also permuted, then all
27 components of the first hyperpolarizability tensor will be computed.

\hypertarget{computational-procedure}{%
\subsection{Computational Procedure}\label{computational-procedure}}

The basic quantities we need are the matrices \(C, \mu, F, \epsilon\),
and \(U\). The MO coefficients \(C\) are already obtained from the
ground-state calculation, along with \(\epsilon^{0}\) (the MO energies)
and \(F^{0}\) (the AO-basis Fock matrix). The dipole matrices \(\mu\)
are needed for the linear response (polarizability) calculation, which
results in response vectors that compose the off-diagonal blocks of the
\(U^{a}\) matrix. \(G^{a}\) is obtained from \(F^{a}\), which comes from
performing a single Fock build with the perturbed density \(D^{a}\).
Finally, \(\epsilon^{a}\) can be constructed.

Although the expressions so far are general for any frequency and
first-order non-linear optical response, the tutorial implementation
will cover the static case. For second-harmonic generation, see the
reference implementation.

\begin{minted}[]{python}
import numpy as np
np.set_printoptions(3, linewidth=100, suppress=True)    # when we inspect the vectors/matrices,
                                                        # use a prettier format for printing
import psi4
\end{minted}

The energy and density convergence criteria are tightened from defaults,
as response properties are sensitive to the quality of the ground-state
wavefunction.

\begin{minted}[]{python}
mol = psi4.geometry('''
    O
    H  1  0.9435
    H  1  0.9435  2  105.9443
    symmetry c1
''')
psi4.set_options({
    "basis": "aug-cc-pVDZ",
    "scf_type": "direct",
    "df_scf_guess": False,
    "e_convergence": 1e-9,
    "d_convergence": 1e-9,
})
\end{minted}

\begin{minted}[]{python}
# This is to enable testing outside of the notebook environment.
import sys
try:
    get_ipython()
    sys.path.append('../../Response-Theory/Self-Consistent-Field')
except NameError:
    import os.path
    dirname = os.path.dirname(os.path.abspath(__file__))
    sys.path.append(os.path.join(dirname, '../../Response-Theory/Self-Consistent-Field'))

from helper_CPHF import helper_CPHF
\end{minted}

The helper encapsulates the solution of the ground-state wavefunction
followed by the frequency-(in)dependent linear response equations,

\[
\left[
\begin{pmatrix}
\mathbf{A} & \mathbf{B} \\ 
\mathbf{B}^{*} & \mathbf{A}^{*}
\end{pmatrix}
- \omega_{f}
\begin{pmatrix}
\mathbf{\Sigma} & \mathbf{\Delta} \\ 
-\mathbf{\Delta}^{*} & -\mathbf{\Sigma}^{*}
\end{pmatrix}
\right]
\begin{pmatrix}
\mathbf{X} \\ 
\mathbf{Y}
\end{pmatrix}
=
\begin{pmatrix}
\mathbf{V} \\ 
-\mathbf{V}^{*}
\end{pmatrix}
,
\]

either directly (via matrix inversion in the MO basis) or iteratively
(via repeated matrix-vector products using Fock builds). For a HF/DFT
reference with canonical orbitals, the above equations reduce to

\[
\left[
\begin{pmatrix}
\mathbf{A} & \mathbf{B} \\ 
\mathbf{B} & \mathbf{A}
\end{pmatrix}
- \omega_{f}
\begin{pmatrix}
\mathbf{1} & \mathbf{0} \\ 
\mathbf{0} & -\mathbf{1}
\end{pmatrix}
\right]
\begin{pmatrix}
\mathbf{X} \\ 
\mathbf{Y}
\end{pmatrix}
=
\begin{pmatrix}
\mathbf{V} \\ 
-\mathbf{V}
\end{pmatrix}
.
\]

In the static limit \((\omega_f = 0)\), the whole superoverlap matrix
vanishes, and the CPHF equations can be reduced to those used in
tutorial 6a.

\begin{minted}[]{python}
solver = helper_CPHF(mol)
solver.run()
\end{minted}

\begin{minted}[]{text}
Number of occupied orbitals: 5
Number of basis functions: 41

Tensor sizes:
ERI tensor           0.02 GB.
oNNN MO tensor       0.00 GB.
ovov Hessian tensor  0.00 GB.

Forming Hessian...
...formed Hessian in 0.473 seconds.

Inverting Hessian...
...inverted Hessian in 0.007 seconds.
\end{minted}

Because the calculation of \(\beta\) requires \(U^{a}\), we also obtain
linear response properties from a quadratic response calculation. This
holds for any order of response, where lower-order response functions
are automatically obtained from higher-order response calculations.

\begin{minted}[]{python}
print(np.around(solver.polar, 4))
\end{minted}

\begin{minted}[]{text}
[[ 7.2587 -0.      0.    ]
 [-0.      8.7969  0.    ]
 [ 0.      0.      7.854 ]]
\end{minted}

\begin{minted}[]{python}
# epsilon^{0}
moenergies = solver.epsilon
C = np.asarray(solver.C)
Co = solver.Co
Cv = solver.Cv
nbf, norb = C.shape
nocc = Co.shape[1]
nvir = norb - nocc
nov = nocc * nvir
# the response vectors X_x, X_y, X_z; Y_x, Y_y, Y_z not needed separately for static response
x = np.asarray(solver.x)
ncomp = x.shape[0]
# reuse the AO-basis dipole integrals
integrals_ao = np.asarray([np.asarray(dipole_ao_component)
                           for dipole_ao_component in solver.tmp_dipoles])
print("dimension of response vectors from linear response: {}".format(x.shape))
# for dynamic response, this will be (2 * nov)
assert x.shape[1] == nov
\end{minted}

\begin{minted}[]{text}
dimension of response vectors from linear response: (3, 180)
\end{minted}

The foundation of the CPHF equations is that the right-hand side
\(\mathbf{V}\) is a perturbation on the wavefunction causing single
excitations from the occupied orbitals to virtual orbitals, the
coefficents of which are in the response vectors \(\mathbf{X}\); the
vectors \(\mathbf{Y}\) describe single deexcitations. Because the full
(square) \(\mathbf{U}\) matrices are required, all MO-based quantities
must be of shape \([N_{orb}, N_{orb}]\) rather than
\([N_{occ}, N_{vir}]\).

\begin{minted}[]{python}
# form full MO-basis dipole integrals
integrals_mo = np.empty(shape=(ncomp, norb, norb))
for i in range(ncomp):
    integrals_mo[i, ...] = (C.T).dot(integrals_ao[i, ...]).dot(C)
\end{minted}

Similarly, \(\mathbf{X}\) and \(\mathbf{Y}\) form the off-diagonal
blocks of the \(\mathbf{U}\) matrices. They are usually stored as in
DALTON, where each vector is of length \(2N_{ov}\), with \(\mathbf{X}\)
on top of \(\mathbf{Y}\).

\begin{minted}[]{python}
# repack response vectors to [norb, norb]; 1/2 is due to X + Y
U = np.zeros_like(integrals_mo)
for i in range(ncomp):
    U[i, :nocc, nocc:] = 0.5 * x[i, ...].reshape(nocc, nvir)
    U[i, nocc:, :nocc] = -0.5 * x[i, ...].reshape(nocc, nvir).T
\end{minted}

A minor implementation detail: because this was not a
frequency-dependent calculation, only \(\mathbf{X}+\mathbf{Y}\) needs to
be calculated; as they are identical, this leads to the prefactor of
1/2.

\begin{minted}[]{python}
# form G matrices from perturbation and generalized Fock matrices; do
# one more Fock build for each response vector
jk = psi4.core.JK.build(solver.scf_wfn.basisset())
jk.initialize()
G = np.empty_like(U)
R = psi4.core.Matrix(nbf, nocc)
npR = np.asarray(R)
for i in range(ncomp):
    V = integrals_mo[i, ...]

    # eqn. (III-1b)
    # Note: this simplified handling of the response vector
    # transformation for the Fock build is insufficient for
    # frequency-dependent response. 1/2 is due to X + Y
    jk.C_clear()
    L = Co
    npR[...] = x[i, ...].reshape(nocc, nvir).dot(np.asarray(Cv).T).T
    jk.C_left_add(L)
    jk.C_right_add(R)
    jk.compute()
    J = 0.5 * np.asarray(jk.J()[0])
    K = 0.5 * np.asarray(jk.K()[0])

    # eqn. (21b)
    F = (C.T).dot(4 * J - K.T - K).dot(C)
    G[i, ...] = V + F

# form epsilon matrices, eqn. (34)
E = G.copy()
omega = 0
for i in range(ncomp):
    eoU = (moenergies[..., np.newaxis] + omega) * U[i, ...]
    Ue = U[i, ...] * moenergies[np.newaxis, ...]
    E[i, ...] += (eoU - Ue)

# Assume some symmetry and calculate only part of the tensor.
# eqn. (VII-4)
hyperpolarizability = np.zeros(shape=(6, 3))
off1 = [0, 1, 2, 0, 0, 1]
off2 = [0, 1, 2, 1, 2, 2]
for r in range(6):
    b = off1[r]
    c = off2[r]
    for a in range(3):
        tl1 = 2 * np.trace(U[a, ...].dot(G[b, ...]).dot(U[c, ...])[:nocc, :nocc])
        tl2 = 2 * np.trace(U[a, ...].dot(G[c, ...]).dot(U[b, ...])[:nocc, :nocc])
        tl3 = 2 * np.trace(U[c, ...].dot(G[a, ...]).dot(U[b, ...])[:nocc, :nocc])
        tr1 = np.trace(U[c, ...].dot(U[b, ...]).dot(E[a, ...])[:nocc, :nocc])
        tr2 = np.trace(U[b, ...].dot(U[c, ...]).dot(E[a, ...])[:nocc, :nocc])
        tr3 = np.trace(U[c, ...].dot(U[a, ...]).dot(E[b, ...])[:nocc, :nocc])
        tr4 = np.trace(U[a, ...].dot(U[c, ...]).dot(E[b, ...])[:nocc, :nocc])
        tr5 = np.trace(U[b, ...].dot(U[a, ...]).dot(E[c, ...])[:nocc, :nocc])
        tr6 = np.trace(U[a, ...].dot(U[b, ...]).dot(E[c, ...])[:nocc, :nocc])
        tl = tl1 + tl2 + tl3
        tr = tr1 + tr2 + tr3 + tr4 + tr5 + tr6
        hyperpolarizability[r, a] = -2 * (tl - tr)
\end{minted}

\begin{minted}[]{python}
ref_static = np.array([
    [ 0.00000001,   0.00000000,  -0.10826460],
    [ 0.00000000,   0.00000000, -11.22412215],
    [ 0.00000000,   0.00000000,  -4.36450397],
    [ 0.00000000,   0.00000000,  -0.00000001],
    [-0.10826460,  -0.00000001,   0.00000000],
    [-0.00000001, -11.22412215,   0.00000000]
])
assert np.allclose(ref_static, hyperpolarizability, rtol=0.0, atol=1.0e-3)
print('\nFirst dipole hyperpolarizability (static):')
print(hyperpolarizability)
\end{minted}

\begin{minted}[]{text}
First dipole hyperpolarizability (static):
[[ -0.       -0.       -0.10826]
 [ -0.       -0.      -11.22412]
 [ -0.       -0.       -4.3645 ]
 [ -0.       -0.        0.     ]
 [ -0.10826   0.       -0.     ]
 [  0.      -11.22412  -0.     ]]
\end{minted}

\hypertarget{references}{%
\subsection{References}\label{references}}

\hypertarget{primary-equations-and-implementation}{%
\subsubsection{Primary equations and
implementation}\label{primary-equations-and-implementation}}

\begin{itemize}
\tightlist
\item
  \href{https://dx.doi.org/10.1002/jcc.540120409}{Karna:1991:487}:
  Karna, Shashi P.; Dupuis, Michel. Frequency dependent nonlinear
  optical properties of molecules: Formulation and implementation in the
  HONDO program \emph{J. Comput. Chem.} \textbf{12} 487-504 (1991)
\end{itemize}

\hypertarget{n-1-rule}{%
\subsubsection{\texorpdfstring{\(2n + 1\)
rule}{2n + 1 rule}}\label{n-1-rule}}

\begin{itemize}
\tightlist
\item
  \href{https://isbnsearch.org/isbn/9780195070286}{Yamaguchi, Yukio;
  Goddard, John D.; Osamura, Yoshihiro; Schaefer III, Henry F. \emph{A
  New Dimension to Quantum Chemistry: Analytic Derivative Methods in Ab
  Initio Molecular Electronic Structure Theory (International Series of
  Monographs on Chemistry);} Oxford University Press: 1994.}
\item
  \href{https://dx.doi.org/10.1063/1.1668053}{Epstein, Saul T. General
  Remainder Theorem \emph{J. Chem. Phys.} \textbf{48} 4725 (1968)}
\item
  \href{https://dx.doi.org/10.1016/0009-2614(80)85340-1}{Epstein, Saul
  T. Constraints and the \(V^{2n+1}\) theorem \emph{Chem. Phys. Lett.}
  \textbf{70} 311 (1980)}
\end{itemize}

\hypertarget{additional-reading}{%
\subsubsection{Additional reading}\label{additional-reading}}

\begin{itemize}
\tightlist
\item
  \href{https://dx.doi.org/10.1016/j.ccr.2008.05.014}{Neese, Frank.
  Prediction of molecular properties and molecular spectroscopy with
  density functional theory: From fundamental theory to
  exchange-coupling \emph{Coord. Chem. Rev.} \textbf{253} 526-563
  (2009)}
\end{itemize}

\end{document}
